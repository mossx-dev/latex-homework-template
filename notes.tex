\documentclass{article}

\usepackage[svgnames]{xcolor}
\usepackage{fancyhdr}
\usepackage{extramarks}
\usepackage{amsmath}
\usepackage{amsthm}
\usepackage{scrextend}
\usepackage{amsfonts}
\usepackage{tikz}
\usepackage{tcolorbox}
\usepackage[plain]{algorithm}
\usepackage{algpseudocode}

\usetikzlibrary{automata,positioning}

%
% Basic Document Settings
%

\topmargin=-0.45in
\evensidemargin=0in
\oddsidemargin=0in
\textwidth=6.5in
\textheight=9.0in
\headsep=0.25in

\linespread{1.1}

\pagestyle{fancy}
\lhead{\authorName}
\chead{\notesClass\ (\classInstructor)}
\rhead{\notesDate}
\lfoot{\lastxmark}
\cfoot{\thepage}

\renewcommand\headrulewidth{0.4pt}
\renewcommand\footrulewidth{0.4pt}

\setlength\parindent{0pt}

%
% Create Problem Sections
%

\newcommand{\enterProblemHeader}[1]{
    \nobreak\extramarks{}{Problem \arabic{#1} continued on next page\ldots}\nobreak{}
    \nobreak\extramarks{Problem \arabic{#1} (continued)}{Problem \arabic{#1} continued on next page\ldots}\nobreak{}
}

\newcommand{\exitProblemHeader}[1]{
    \nobreak\extramarks{Problem \arabic{#1} (continued)}{Problem \arabic{#1} continued on next page\ldots}\nobreak{}
    \stepcounter{#1}
    \nobreak\extramarks{Problem \arabic{#1}}{}\nobreak{}
}

\setcounter{secnumdepth}{0}
\newcounter{partCounter}
\newcounter{homeworkProblemCounter}
\setcounter{homeworkProblemCounter}{1}
\nobreak\extramarks{Problem \arabic{homeworkProblemCounter}}{}\nobreak{}

%
% Homework Problem Environment
%
% This environment takes an optional argument. When given, it will adjust the
% problem counter. This is useful for when the problems given for your
% assignment aren't sequential. See the last 3 problems of this template for an
% example.
%
\newenvironment{homeworkProblem}[1][-1]{
    \ifnum#1>0
        \setcounter{homeworkProblemCounter}{#1}
    \fi
    \section{Problem \arabic{homeworkProblemCounter}}
    \setcounter{partCounter}{1}
    \enterProblemHeader{homeworkProblemCounter}
}{
    \exitProblemHeader{homeworkProblemCounter}
}

\newtcolorbox{definition}{colframe=DarkViolet!40!black,colback=LightSlateGray!15!white,fonttitle=\bfseries,title=Definition}

\newtcolorbox{example}[1]{colframe=DarkSlateGray!75!black,colback=LightSlateGray!15!white,fonttitle=\bfseries,title=Example: {#1}}

\newtcolorbox{subproblem}[1]{colframe=DarkGreen!65!black,colback=LightSlateGray!15!white,fonttitle=\bfseries,title={#1}}

%
% Homework Details
%   - Title
%   - Due date
%   - Class
%   - Section/Time
%   - Instructor
%   - Author
%

\newcommand{\notesDate}{November 9, 2022}
\newcommand{\notesClass}{Calculus 3}
% \newcommand{\hmwkClassTime}{Section 4.1}
\newcommand{\classInstructor}{Erin Craig}
\newcommand{\authorName}{\textbf{Moss Gallagher}}

%
% Title Page
%

\title{
    \vspace{2in}
    \textmd{\textbf{\notesClass}}\\
    \normalsize\vspace{0.1in}\small{\notesDate}\\
    \vspace{0.1in}\large{\textit{\classInstructor}}
    \vspace{3in}
}

\author{\authorName}
\date{}

\renewcommand{\part}[1]{\textbf{\large Part \Alph{partCounter}}\stepcounter{partCounter}\\}

%
% Various Helper Commands
%

% Useful for quickly taking notes
\newcommand{\newsection}[3][2]{\section{Section #2: #3}}
% \newcommand{\example}[1]{\subsection{Example: #1}}
% \newcommand{\subproblem}[1]{\quad \quad \large{#1}}

% Useful for algorithms
\newcommand{\alg}[1]{\textsc{\bfseries \footnotesize #1}}

% For derivatives
\newcommand{\deriv}[1]{\frac{\mathrm{d}}{\mathrm{d}x} (#1)}

% For partial derivatives
\newcommand{\pderiv}[2]{\frac{\partial}{\partial #1} (#2)}

% Integral dx
\newcommand{\dx}{\mathrm{d}x}

% Alias for the Solution section header
\newcommand{\solution}{\textbf{\large Solution}}

% Probability commands: Expectation, Variance, Covariance, Bias
\newcommand{\E}{\mathrm{E}}
\newcommand{\Var}{\mathrm{Var}}
\newcommand{\Cov}{\mathrm{Cov}}
\newcommand{\Bias}{\mathrm{Bias}}

\begin{document}

\maketitle

\pagebreak

\newsection{4.1}{Functions of Several Variables}

\begin{definition}
	A \textbf{vector-valued function} is a function of the form
	$$r(t) = f(t)i + g(t)j \quad or \quad r(t) = f(t)i + g(t)j + h(t)k$$
	where the \textbf{component functions} $f$. $g$. and $h$, are real-valued functions of the parameter t. Vector-valued functions are also written in the form
	$$r(t) = \left< f(t), g(t) \right> \quad or \quad r(t) = \left< f(t), g(t), h(t) \right>$$
	In both cases, the first form of the function defines a two-dimensional vector valued function; the second form describes a three-dimensional vector valued function.
\end{definition}

\begin{example}{Evaluate the functions at $r(0)$, $r\left(\frac{\pi}{2}\right)$, $r\left(\frac{2\pi}{3}\right)$}

	\begin{subproblem}{$r(t)$ = $\left<t, 1\right>$}
		\textbf{At $r\left(0\right)$}
		\[\begin{split}
			r(0) &= \left<0, 1\right>
		\end{split}\] \\
		\textbf{At $r\left(\frac{\pi}{2}\right)$}
		\[
			\begin{split}
			r(\frac{\pi}{2}) &= \left<\frac{\pi}{2}, 1\right>
			\end{split}
		\] \\
		\textbf{At $r\left(\frac{2\pi}{3}\right)$}
		\[
			\begin{split}
			r(\frac{2\pi}{3}) &= \left<\frac{2\pi}{3}, 1\right>
			\end{split}
		\]
	\end{subproblem}

	\begin{subproblem}{$r(t)$ = $\left<t^2, t\right>$}
		\textbf{At $r\left(0\right)$}
		\[\begin{split}
			r(0) &= \left<0^2, 0\right> \\
			r(0) &= \left<0, 0\right>
		\end{split}\] \\
		\textbf{At $r\left(\frac{\pi}{2}\right)$}
		\[
			\begin{split}
			r(\frac{\pi}{2}) &= \left<\frac{\pi}{2}^2, \frac{\pi}{2}\right> \\
			r(\frac{\pi}{2}) &= \left<\frac{\pi^2}{4}, \frac{\pi}{2}\right>
			\end{split}
		\] \\
		\textbf{At $r\left(\frac{2\pi}{3}\right)$}
		\[
			\begin{split}
			r(\frac{2\pi}{3}) &= \left<\frac{2\pi}{3}^2, \frac{2\pi}{3}\right> \\
			r(\frac{2\pi}{3}) &= \left<\frac{4\pi^2}{9}, \frac{2\pi}{3}\right>
			\end{split}
		\]
	\end{subproblem}

\end{example}

\end{document}
